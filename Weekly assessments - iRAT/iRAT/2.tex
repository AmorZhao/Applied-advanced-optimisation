% Reminder:
% - Watch the recordings in Chapter 2 and Chapter 3 up to Section 3.4 (included)
% - Complete the second iRAT by Thursday 23rd at 13:00
% - Next class on Friday 24th at 11:00.

\subsection*{Question 1}
\textbf{Define a randomized detector and a deterministic detector.}

A randomized detector $T$ is $t_{ik} = \textbf{prob}(\hat \theta = i | x = k)$, where, if we observe $x=k$, then the detector returns the hypothesis $\hat \theta = i$ with probability $t_{ik}$. 

A deterministic detector is a detector whose behavior does not involve any randomness, ie. always gives the same result if the same input is processed. So for a deterministic detector, $t_{ik} = 1$ if $\hat \theta = i$ and $0$ otherwise.


\subsection*{Question 2}
\textbf{Define the detection probability matrix.}

The detection probability matrix can be defined as $D = TP$. 

$$
d_{ij} = (TP)_{ij} = \textbf{prob}(\hat \theta = i | \theta = j).
$$

\subsection*{Question 3}
\textbf{Define a convex combination and use it to define a convex set.}

A convex combination of the points $x_1, \dots, x_k \in C$ is a point of the form $\theta_1 x_1 + \cdots+ \theta_k x_k$, with $\theta_1 + \cdots+\theta_k = 1$ and $\theta_i\ge 0$, for all $i = 1,\dots, k$ (ie. a linear combination \( \sum_{i=1}^k \lambda_i x_i \) where each \( \lambda_i \geq 0 \) and \( \sum_{i=1}^k \lambda_i = 1 \)).

A set $C$ is a convex set if and only if it contains every convex combination of its points. (ie. ``every point can be seen by every other point in the set")

\subsection*{Question 4}
\textbf{Make a sketch of a convex set with exactly two corners (this is not in the handouts, think).}

The shaded region in the figure below: 

\begin{figure}[h]
\centering
\includegraphics[width=0.5\textwidth]{iRAT/Images/image_2_1.png}
\caption{A convex set with exactly two corners (moon shape)}
\end{figure}

\subsection*{Question 5}
\textbf{Explain the difference between an affine set, a convex set and a conic set.}

An affine set is closed under affine combinations: if every affine combination $\theta_1 x_1 + \cdots+ \theta_k x_k$, with $\theta_1 + \cdots+\theta_k = 1$, of its points  $x_1, \dots, x_k \in C$ belongs to $C$. 

A convex set is closed under concex combinations: it requires that for any \( x, y \in C \) and \( \lambda \in [0,1] \), \( \lambda x + (1-\lambda)y \in C \). This ensures the line segment between any two points is entirely within the set.

A conic set is closed under positive scalar multiplication: for any \( x \in C \) and \( \alpha > 0 \), \( \alpha x \in C \). This makes it a ray of cone region from the origin. 

\subsection*{Question 6}
\textbf{Prove that the positive semidefinite cone is a convex cone using the definition of convex cone (as done in the video of Section 3.3).}

A set \( C \) is a convex cone if for any \( A, B \in C \) and any non-negative scalars \( \alpha, \beta \geq 0 \), the combination \( \alpha A + \beta B \in C \).

Let \( A \) and \( B \) be positive semidefinite matrices, i.e., \( A \succeq 0 \) and \( B \succeq 0 \). Let \( \alpha, \beta \geq 0 \).

Since for any vector \( x \in \mathbb{R}^n \),
\[
x^T (\alpha A + \beta B) x = \alpha x^T A x + \beta x^T B x \geq 0,
\]
givent that \( A \succeq 0 \) and \( B \succeq 0 \).

Therefore \( \alpha A + \beta B \) is positive semidefinite. Hence the positive semidefinite cone is a convex cone.

\subsection*{Question 7}
\textbf{Prove that the positive semidefinite cone is convex using the intersection property (as done in Section 3.4).}

A matrix \( A \in \mathbb{R}^{n \times n} \) is positive semidefinite if and only if for all vectors \( x \in \mathbb{R}^n \), \( x^T A x \geq 0 \).

This condition can be expressed as an intersection of convex sets. For each \( x \in \mathbb{R}^n \), define the set:
\[
C_x = \{ A \in \mathbb{R}^{n \times n} \mid x^T A x \geq 0 \}
\]
Each \( C_x \) is a convex set because it is defined by a linear inequality in terms of \( A \). The positive semidefinite cone \( S_+^n \) is the intersection of all such sets \( C_x \). Since convexity is preserved under intersection, \( S_+^n \) is convex. Hence the positive semidefinite cone is convex. 

\newpage

\section*{Feeback}

\subsection*{Q4}

One student generated a figure using AI that looks like a triangle with the left side curved in. This is obviously incorrect because the side curves in and if you track a line between any two points on this curve, that line would lies outside the set. Guys, AI is rubbish at these things, I try it often and I am always disappointed. 

One student misunderstood the meaning of corner, thinking that a corner is any point on the border of the set (so a circle would be full of corners). This is not the case. A corner is a sharp intersection of two edges. Call it a vertex, or a point where the edge is not differentiable, or a point which cannot be obtained as a convex combination of any other point but itself.

It was nice to see a lot of different sets here. Many proposed "-", "D" "()" or infinitely long half rectanles.

\subsection*{Q5}

Here you gave a lot of different interpretations and observations. I generally considered correct almost all of them, but I want to make two remarks: 

\begin{itemize}
    \item The standard answer should include the differences on the assumptions on the weights (sum to 1 and/or non-negativity). 
    \item There was one answer which I considered wrong: a student wrote that affine sets are one-dimensional because they are lines. This is incorrect because affine sets are not lines, but sets that contain any line passing through any two points belonging to the set. Affine sets can be of any dimension. $R^2$ (the entire plane) is an affine set and it is an example of two-dimensional affine set.
\end{itemize}
