% For next week:
% - Cover from section 6.3 to section 7.3.2 (included).
% - Complete the sixth iRAT by Thursday 20th at 13:00
% - Next class on Friday 21th at 11:00 in 305.

\subsection*{Question 1}
\textbf{Let $p^*(u, v)$ the optimal value of a perturbed problem. Assume $p^*(u, v)$ is differentiable at $u = 0$, $v = 0$ and that strong duality holds.}

\noindent \textbf{Provide the relation between the optimal dual variables and the optimal value of the perturbed problem.} 

\noindent \textbf{Using these relations explain what happens when tightening or loosening the i-th inequality constraint by a small amount.}



\subsection*{Question 2}
\textbf{Explain the factor-solve method.}

\subsection*{Question 3}
\textbf{Explain why sparsity is important when solving linear equations.}

\subsection*{Question 4}
\textbf{Provide a bound on $f (x) - p^*$ using strong convexity.}

\subsection*{Question 5}
\textbf{Using classical convergence analysis provide an upper bound on the number of iterations until $f (x^{(k)}) - p^* \leq \epsilon$.}




\section*{Appendix}
\noindent\textbf{1. \; Write the Karush-Kuhn-Tucker (KKT) conditions.}


\noindent\textbf{2. \; Explain why it may be useful to introduce new variables in a problem in the context of duality.}


\noindent\textbf{3. \; Define strong convexity.}




\noindent\textbf{4. \; Write the Netwon step and the Newton decrement.}





